\section{Технологическая часть}

В данном разделе рассматриваются выбранные средства реализации программного обеспечения.
Также представлены листинги исходного кода и модули разработанного сервера.
Реализация учитывает формализованные требования к разрабатываемому серверу и особенности задания на данную работу.
Например, поддерживается технология пул потоков, а в основном цикле программы происходит обращение к системному вызову select.

\subsection{Средства реализации и модули программы}

В качестве языка программирования, используемого при реализации, выбран язык Си.

Причины выбора языка:
\begin{itemize}
	\item наличие системного вызова select;
	\item соответствие требованиям задания на данную работу;
	\item наличие системных вызовов сетевого стека для работы с сокетами (socket, accept, listen и так далее).
\end{itemize}

Программа представлена следующими модулями:
\begin{enumerate}
	\item main (содержит точку входа в программу, цикл с select);
	\item thread\_pool (содержит функцию создания пула потоков);
	\item logger (содержит функции записи информации о событиях в программе);
	\item server (содержит функции обработки запросов);
	\item http (содержит константы HTTP протокола).
\end{enumerate}

Получение исполняемого файла происходит при помощи компилятора gcc.

Сборка проекта происходит с использованием утилиты make.
Пример правила из файла Makefile приведен на листинге \ref{lst:make}.

\begin{listing}[!h]
	\caption{Правило make}
	\label{lst:make}
	\begin{minted}[frame=single,fontsize = \footnotesize, linenos, xleftmargin = 1.5em]{c}
server: out/main.o out/logger.o out/thread_pool.o out/server.o
	$(CC) -o $@ $^ -pthread

out/logger.o: src/logger.c inc/logger.h
	$(CC) $(CFLAGS) -o $@ -c $<
	\end{minted}
\end{listing}

\subsection{Реализация сервера}

На листинге \ref{lst:main1} представлена первая часть куска реализации точки входа в программу с циклом обработки запросов, который содержит select.

\begin{listing}[!htb]
	\caption{Часть реализации точки входа в программу с основным циклом (часть 1)}
	\inputminted[frame=single,fontsize = \footnotesize, linenos, breaklines, xleftmargin = 1.5em,breaksymbol = ""]{c}{../lst/main.c}
	\label{lst:main1}
\end{listing}

\newpage

На листинге \ref{lst:main2} представлена вторая часть куска реализации точки входа в программу.
Реализация функции (часть 1) работы потока из пула представлена на листинге~\ref{lst:thread}.

\begin{listing}[!htb]
\caption{Часть реализации точки входа в программу с основным циклом (часть 2)}
\inputminted[frame=single,fontsize = \footnotesize, linenos, breaklines, xleftmargin = 1.5em,breaksymbol = ""]{c}{../lst/main2.c}
\label{lst:main2}
\end{listing}

\begin{listing}[!htb]
	\caption{Часть реализации функции работы потока из пула (часть 1)}
	\inputminted[frame=single,fontsize = \footnotesize, linenos, breaklines, xleftmargin = 1.5em,breaksymbol = ""]{c}{../lst/thread.c}
	\label{lst:thread}
\end{listing}

Реализация функции (часть 2) работы потока из пула представлена на листинге \ref{lst:thread2}.

\begin{listing}[!htb]
\caption{Часть реализации функции работы потока из пула (часть 2)}
\inputminted[frame=single,fontsize = \footnotesize, linenos, breaklines, xleftmargin = 1.5em,breaksymbol = ""]{c}{../lst/thread2.c}
\label{lst:thread2}
\end{listing}