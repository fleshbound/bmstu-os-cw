\section{Технологический раздел}

\subsection{Выбор языка и среды программирования}

В качестве языка программирования выбран язык C \cite{gnu} стандарта C99, так как на этом языке написан код ядра операционной системы Linux.

Для компиляции модуля был использован GNU Compiler Collection (GCC)~\cite{gcc} --- стандартный компилятор UNIX-подобных операционных систем.

Сборка модуля драйвера производится автоматически с помощью утилиты make~\cite{make}.

\subsection{Реализация драйвера}

На листинге \ref{lst:makefile} представлен файл сборки модуля драйвера Makefile, используемый утилитой make.

\begin{longlisting}
	\singlespacing
	\caption{Файл сборки модуля драйвера Makefile}
	\label{lst:makefile}
	\begin{minted}[frame=single,fontsize = \footnotesize, linenos, xleftmargin = 1.5em]{c}
ifneq ($(KERNELRELEASE),)
  obj-m := aceline.o
else
  CURRENT = $(shell uname -r)
  KDIR = /lib/modules/$(CURRENT)/build 
  PWD = $(shell pwd)

default:
  echo $(MAKE) -C $(KDIR) M=$(PWD) modules
  $(MAKE) -C $(KDIR) M=$(PWD) modules
  make clean

clean:
  @rm -f *.o .*.cmd .*.flags *.mod.c *.order
  @rm -f .*.*.cmd *~ *.*~ TODO.*
  @rm -fR .tmp*
  @rm -rf .tmp_versions

disclean: clean
  @rm *.ko *.symvers
  
endif
	\end{minted}
\end{longlisting}

Реализация драйвера представлена на листинге \ref{lst:driver}.

\begin{longlisting}
	\singlespacing
	\caption{Реализация драйвера}
	\inputminted[frame=single,fontsize = \footnotesize, linenos, breaklines, xleftmargin = 1.5em,breaksymbol = ""]{c}{../lst/driver.c}
	\label{lst:driver}
\end{longlisting}

\subsection{Реализация демона}

Реализация демона представлена на листинге \ref{lst:daemon}.

\begin{longlisting}
	\singlespacing
	\caption{Реализация демона}
	\inputminted[frame=single,fontsize = \footnotesize, linenos, breaklines, xleftmargin = 1.5em,breaksymbol = ""]{c}{../lst/daemon.c}
	\label{lst:daemon}
\end{longlisting}
