\let\counterwithout\relax
\let\counterwithin\relax
%\usepackage{lmodern}
\usepackage{float}
\usepackage{xcolor}
\usepackage{extsizes}
\usepackage{subfig}
\usepackage[export]{adjustbox}
\usepackage{tocvsec2} % возможность менять учитываемую глубину разделов в оглавлении
\usepackage[subfigure]{tocloft}
\usepackage[newfloat]{minted}
\captionsetup[listing]{position=top}

\AtBeginEnvironment{figure}{\vspace{0.5cm}}
\AtBeginEnvironment{table}{\vspace{0.5cm}}
\AtBeginEnvironment{listing}{\vspace{0.5cm}}
\AtBeginEnvironment{algorithm}{\vspace{0.5cm}}
\AtBeginEnvironment{minted}{\vspace{-0.5cm}}

\usepackage{fancyvrb}
\usepackage{ulem,bm,mathrsfs,ifsym} %зачеркивания, особо жирный стиль и RSFS начертание
\usepackage{sectsty} % переопределение стилей подразделов
%%%%%%%%%%%%%%%%%%%%%%%

%%% Поля и разметка страницы %%%
\usepackage{pdflscape}                              % Для включения альбомных страниц
\usepackage{geometry}                               % Для последующего задания полей
\geometry{a4paper,tmargin=2cm,bmargin=2cm,lmargin=3cm,rmargin=1cm} % тоже самое, но лучше

%%% Математические пакеты %%%
\usepackage{amsthm,amsfonts,amsmath,amssymb,amscd}  % Математические дополнения от AMS
\usepackage{mathtools}                              % Добавляет окружение multlined
\usepackage[perpage]{footmisc}
%\usepackage{times}

%%%% Установки для размера шрифта 14 pt %%%%
%% Формирование переменных и констант для сравнения (один раз для всех подключаемых файлов)%%
%% должно располагаться до вызова пакета fontspec или polyglossia, потому что они сбивают его работу
%\newlength{\curtextsize}
%\newlength{\bigtextsize}
%\setlength{\bigtextsize}{13pt}
\KOMAoptions{fontsize=14pt}

\makeatletter
\def\showfontsize{\f@size{} point}
\makeatother

\makeatletter
\setlength{\@fptop}{0pt}
\makeatother

%\makeatletter
%\show\f@size                                       % неплохо для отслеживания, но вызывает стопорение процесса, если документ компилируется без команды  -interaction=nonstopmode 
%\setlength{\curtextsize}{\f@size pt}
%\makeatother

%шрифт times
\usepackage{tempora}
%\usepackage{pscyr}
%\setmainfont[Ligatures={TeX,Historic}]{Times New Roman}

%%% Решение проблемы копирования текста в буфер кракозябрами
%    \input glyphtounicode.tex
%    \input glyphtounicode-cmr.tex %from pdfx package
%    \pdfgentounicode=1
\usepackage{cmap}                               % Улучшенный поиск русских слов в полученном pdf-файле
\usepackage[T1]{fontenc}                       % Поддержка русских букв
\usepackage[utf8]{inputenc}                     % Кодировка utf8
\usepackage[english, main=russian]{babel}            % Языки: русский, английский
%   \IfFileExists{pscyr.sty}{\usepackage{pscyr}}{}  % Красивые русские шрифты
%\renewcommand{\rmdefault}{ftm}
%%% Оформление абзацев %%%
\usepackage{indentfirst}                            % Красная строка
%\usepackage{eskdpz}

%%% Таблицы %%%
\usepackage{longtable}                              % Длинные таблицы
\usepackage{multirow,makecell,array}                % Улучшенное форматирование таблиц
\usepackage{booktabs}                               % Возможность оформления таблиц в классическом книжном стиле (при правильном использовании не противоречит ГОСТ)

%%% Общее форматирование
\usepackage{soulutf8}                               % Поддержка переносоустойчивых подчёркиваний и зачёркиваний
\usepackage{icomma}                                 % Запятая в десятичных дробях
\usepackage{siunitx}


%%% Изображения %%%
\usepackage{graphicx}                               % Подключаем пакет работы с графикой
\usepackage{wrapfig}

%%% Списки %%%
\usepackage{enumitem}

%%% Подписи %%%
\usepackage{caption}                                % Для управления подписями (рисунков и таблиц) % Может управлять номерами рисунков и таблиц с caption %Иногда может управлять заголовками в списках рисунков и таблиц
%% Использование:
%\begin{table}[h!]\ContinuedFloat - чтобы не переключать счетчик
%\captionsetup{labelformat=continued}% должен стоять до самого caption
%\caption{}
% либо ручками \caption*{Продолжение таблицы~\ref{...}.} :)

%%% Интервалы %%%
\addto\captionsrussian{%
	\renewcommand{\listingname}{Листинг}%
}
%%% Счётчики %%%
\usepackage[figure,table,section]{totalcount}               % Счётчик рисунков и таблиц
\DeclareTotalCounter{lstlisting}
\usepackage{totcount}                               % Пакет создания счётчиков на основе последнего номера подсчитываемого элемента (может требовать дважды компилировать документ)
\usepackage{totpages}                               % Счётчик страниц, совместимый с hyperref (ссылается на номер последней страницы). Желательно ставить последним пакетом в преамбуле

%%% Продвинутое управление групповыми ссылками (пока только формулами) %%%
%% Кодировки и шрифты %%%

%   \newfontfamily{\cyrillicfont}{Times New Roman}
%   \newfontfamily{\cyrillicfonttt}{CMU Typewriter Text}
%\setmainfont{Times New Roman}
%\newfontfamily\cyrillicfont{Times New Roman}
%\setsansfont{Times New Roman}                    %% задаёт шрифт без засечек
%	\setmonofont{Liberation Mono}               %% задаёт моноширинный шрифт
%    \IfFileExists{pscyr.sty}{\renewcommand{\rmdefault}{ftm}}{}
%%% Интервалы %%%
%linespread-реализация ближе к реализации полуторного интервала в ворде.
%setspace реализация заточена под шрифты 10, 11, 12pt, под остальные кегли хуже, но всё же ближе к типографской классике. 
\linespread{1.3}                    % Полуторный интервал (ГОСТ Р 7.0.11-2011, 5.3.6)
%\renewcommand{\@biblabel}[1]{#1}

%%% Гиперссылки %%%
\usepackage{hyperref}

%%% Выравнивание и переносы %%%
\sloppy                             % Избавляемся от переполнений
\clubpenalty=10000                  % Запрещаем разрыв страницы после первой строки абзаца
\widowpenalty=10000                 % Запрещаем разрыв страницы после последней строки абзаца

\makeatletter % малые заглавные, small caps shape
\let\@@scshape=\scshape
\renewcommand{\scshape}{%
	\ifnum\strcmp{\f@series}{bx}=\z@
	\usefont{T1}{cmr}{bx}{sc}%
	\else
	\ifnum\strcmp{\f@shape}{it}=\z@
	\fontshape{scsl}\selectfont
	\else
	\@@scshape
	\fi
	\fi}
\makeatother

%%% Подписи %%%
%\captionsetup{%
	%singlelinecheck=off,                % Многострочные подписи, например у таблиц
	%skip=2pt,                           % Вертикальная отбивка между подписью и содержимым рисунка или таблицы определяется ключом
	%justification=centering,            % Центрирование подписей, заданных командой \caption
	%}
%%%        Подключение пакетов                 %%%
\usepackage{ifthen}                 % добавляет ifthenelse
%%% Инициализирование переменных, не трогать!  %%%
\newcounter{intvl}
\newcounter{otstup}
\newcounter{contnumeq}
\newcounter{contnumfig}
\newcounter{contnumtab}
\newcounter{pgnum}
\newcounter{bibliosel}
\newcounter{chapstyle}
\newcounter{headingdelim}
\newcounter{headingalign}
\newcounter{headingsize}
\newcounter{tabcap}
\newcounter{tablaba}
\newcounter{tabtita}
%%%%%%%%%%%%%%%%%%%%%%%%%%%%%%%%%%%%%%%%%%%%%%%%%%

%%% Область упрощённого управления оформлением %%%

%% Интервал между заголовками и между заголовком и текстом
% Заголовки отделяют от текста сверху и снизу тремя интервалами (ГОСТ Р 7.0.11-2011, 5.3.5)
\setcounter{intvl}{3}               % Коэффициент кратности к размеру шрифта

%% Отступы у заголовков в тексте
\setcounter{otstup}{0}              % 0 --- без отступа; 1 --- абзацный отступ

%% Нумерация формул, таблиц и рисунков
\setcounter{contnumeq}{1}           % Нумерация формул: 0 --- пораздельно (во введении подряд, без номера раздела); 1 --- сквозная нумерация по всей диссертации
\setcounter{contnumfig}{1}          % Нумерация рисунков: 0 --- пораздельно (во введении подряд, без номера раздела); 1 --- сквозная нумерация по всей диссертации
\setcounter{contnumtab}{1}          % Нумерация таблиц: 0 --- пораздельно (во введении подряд, без номера раздела); 1 --- сквозная нумерация по всей диссертации

%% Оглавление
\setcounter{pgnum}{0}               % 0 --- номера страниц никак не обозначены; 1 --- Стр. над номерами страниц (дважды компилировать после изменения)

%% Библиография
\setcounter{bibliosel}{1}           % 0 --- встроенная реализация с загрузкой файла через движок bibtex8; 1 --- реализация пакетом biblatex через движок biber

%% Текст и форматирование заголовков
\setcounter{chapstyle}{1}           % 0 --- разделы только под номером; 1 --- разделы с названием "Глава" перед номером
\setcounter{headingdelim}{1}        % 0 --- номер отделен пропуском в 1em или \quad; 1 --- номера разделов и приложений отделены точкой с пробелом, подразделы пропуском без точки; 2 --- номера разделов, подразделов и приложений отделены точкой с пробелом.

%% Выравнивание заголовков в тексте
\setcounter{headingalign}{0}        % 0 --- по центру; 1 --- по левому краю

%% Размеры заголовков в тексте
\setcounter{headingsize}{0}         % 0 --- по ГОСТ, все всегда 14 пт; 1 --- пропорционально изменяющийся размер в зависимости от базового шрифта

%% Подпись таблиц
\setcounter{tabcap}{0}              % 0 --- по ГОСТ, номер таблицы и название разделены тире, выровнены по левому краю, при необходимости на нескольких строках; 1 --- подпись таблицы не по ГОСТ, на двух и более строках, дальнейшие настройки: 
%Выравнивание первой строки, с подписью и номером
\setcounter{tablaba}{2}             % 0 --- по левому краю; 1 --- по центру; 2 --- по правому краю
%Выравнивание строк с самим названием таблицы
\setcounter{tabtita}{1}             % 0 --- по левому краю; 1 --- по центру; 2 --- по правому краю

%%% Рисунки %%%
\DeclareCaptionLabelSeparator*{emdash}{~--- }             % (ГОСТ 2.105, 4.3.1)
\captionsetup[figure]{labelsep=emdash,font=onehalfspacing,position=bottom}
\captionsetup[listing]{labelsep=emdash,font=onehalfspacing,position=top}

%%% Таблицы %%%
\ifthenelse{\equal{\thetabcap}{0}}{%
	\newcommand{\tabcapalign}{\raggedright}  % по левому краю страницы или аналога parbox
}

\ifthenelse{\equal{\thetablaba}{0} \AND \equal{\thetabcap}{1}}{%
	\newcommand{\tabcapalign}{\raggedright}  % по левому краю страницы или аналога parbox
}

\ifthenelse{\equal{\thetablaba}{1} \AND \equal{\thetabcap}{1}}{%
	\newcommand{\tabcapalign}{\centering}    % по центру страницы или аналога parbox
}

\ifthenelse{\equal{\thetablaba}{2} \AND \equal{\thetabcap}{1}}{%
	\newcommand{\tabcapalign}{\raggedleft}   % по правому краю страницы или аналога parbox
}

\ifthenelse{\equal{\thetabtita}{0} \AND \equal{\thetabcap}{1}}{%
	\newcommand{\tabtitalign}{\raggedright}  % по левому краю страницы или аналога parbox
}

\ifthenelse{\equal{\thetabtita}{1} \AND \equal{\thetabcap}{1}}{%
	\newcommand{\tabtitalign}{\centering}    % по центру страницы или аналога parbox
}

\ifthenelse{\equal{\thetabtita}{2} \AND \equal{\thetabcap}{1}}{%
	\newcommand{\tabtitalign}{\raggedleft}   % по правому краю страницы или аналога parbox
}

\DeclareCaptionFormat{tablenocaption}{\tabcapalign #1\strut}        % Наименование таблицы отсутствует
\ifthenelse{\equal{\thetabcap}{0}}{%
	\DeclareCaptionFormat{tablecaption}{\tabcapalign #1#2#3}
	\captionsetup[table]{labelsep=emdash}                       % тире как разделитель идентификатора с номером от наименования
}{%
	\DeclareCaptionFormat{tablecaption}{\tabcapalign #1#2\par%  % Идентификатор таблицы на отдельной строке
		\tabtitalign{#3}}                                       % Наименование таблицы строкой ниже
	\captionsetup[table]{labelsep=space}                        % пробельный разделитель идентификатора с номером от наименования
}
\captionsetup[table]{format=tablecaption,singlelinecheck=off,font=onehalfspacing,position=top,skip=-5pt}  % многострочные наименования и прочее
\DeclareCaptionLabelFormat{continued}{Продолжение таблицы~#2}
\setlength{\belowcaptionskip}{.2cm}
\setlength{\intextsep}{0ex}

%%% Подписи подрисунков %%%
\renewcommand{\thesubfigure}{\asbuk{subfigure}}           % Буквенные номера подрисунков
\captionsetup[subfigure]{font={normalsize},               % Шрифт подписи названий подрисунков (не отличается от основного)
	labelformat=brace,                                    % Формат обозначения подрисунка
	justification=centering,                              % Выключка подписей (форматирование), один из вариантов            
}
%\DeclareCaptionFont{font12pt}{\fontsize{12pt}{13pt}\selectfont} % объявляем шрифт 12pt для использования в подписях, тут же надо интерлиньяж объявлять, если не наследуется
%\captionsetup[subfigure]{font={font12pt}}                 % Шрифт подписи названий подрисунков (всегда 12pt)

%%% Настройки гиперссылок %%%

\definecolor{linkcolor}{rgb}{0.0,0,0}
\definecolor{citecolor}{rgb}{0,0.0,0}
\definecolor{urlcolor}{rgb}{0,0,0}

\hypersetup{
	linktocpage=true,           % ссылки с номера страницы в оглавлении, списке таблиц и списке рисунков
	%    linktoc=all,                % both the section and page part are links
	%    pdfpagelabels=false,        % set PDF page labels (true|false)
	plainpages=true,           % Forces page anchors to be named by the Arabic form  of the page number, rather than the formatted form
	colorlinks,                 % ссылки отображаются раскрашенным текстом, а не раскрашенным прямоугольником, вокруг текста
	linkcolor={linkcolor},      % цвет ссылок типа ref, eqref и подобных
	citecolor={citecolor},      % цвет ссылок-цитат
	urlcolor={urlcolor},        % цвет гиперссылок
	pdflang={ru},
}
\urlstyle{same}
%%% Шаблон %%%
%\DeclareRobustCommand{\todo}{\textcolor{red}}       % решаем проблему превращения названия цвета в результате \MakeUppercase, http://tex.stackexchange.com/a/187930/79756 , \DeclareRobustCommand protects \todo from expanding inside \MakeUppercase
\setlength{\parindent}{2.5em}                       % Абзацный отступ. Должен быть одинаковым по всему тексту и равен пяти знакам (ГОСТ Р 7.0.11-2011, 5.3.7).

%%% Списки %%%
% Используем дефис для ненумерованных списков (ГОСТ 2.105-95, 4.1.7)
%\renewcommand{\labelitemi}{\normalfont\bfseries~{---}} 
\renewcommand{\labelitemi}{\bfseries~{---}} 
\setlist{nosep,%                                    % Единый стиль для всех списков (пакет enumitem), без дополнительных интервалов.
	labelindent=\parindent,leftmargin=*%            % Каждый пункт, подпункт и перечисление записывают с абзацного отступа (ГОСТ 2.105-95, 4.1.8)
}
%%%%%%%%%%%%%%%%%%%%%%
%\usepackage{xltxtra} % load xunicode

\usepackage{ragged2e}
\usepackage[explicit]{titlesec}
\usepackage{placeins}
\usepackage{xparse}
\usepackage{csquotes}

\usepackage{listingsutf8}
\usepackage{url} %пакеты расширений
\usepackage{algorithm, algorithmicx}
\usepackage[noend]{algpseudocode}
\usepackage{blkarray}
\usepackage{chngcntr}
\usepackage{tabularx}
\usepackage[backend=biber, 
bibstyle=gost-numeric,
citestyle=nature]{biblatex}
\newcommand*\template[1]{\text{<}#1\text{>}}
\addbibresource{biblio.bib}

\titleformat{name=\section,numberless}[block]{\normalfont\Large\centering}{}{0em}{#1}
\titleformat{\section}[block]{\normalfont\Large\bfseries\raggedright}{}{0em}{\thesection\hspace{0.25em}#1}
\titleformat{\subsection}[block]{\normalfont\Large\bfseries\raggedright}{}{0em}{\thesubsection\hspace{0.25em}#1}
\titleformat{\subsubsection}[block]{\normalfont\large\bfseries\raggedright}{}{0em}{\thesubsubsection\hspace{0.25em}#1}
\titleformat{name=\subsubsection,numberless}[block]{\normalfont\large\bfseries\raggedright}{}{0em}{#1}
\titleformat{name=\subsection,numberless}[block]{\normalfont\Large\bfseries\raggedright}{}{0em}{#1}

\let\Algorithm\algorithm
\renewcommand\algorithm[1][]{\Algorithm[#1]\setstretch{1.5}}
%\renewcommand{\listingscaption}{Листинг}

\usepackage{pifont}
\usepackage{calc}
\usepackage{suffix}
\usepackage{csquotes}
\DeclareQuoteStyle{russian}
{\guillemotleft}{\guillemotright}[0.025em]
{\quotedblbase}{\textquotedblleft}
\ExecuteQuoteOptions{style=russian}
\newcommand{\enq}[1]{\enquote{#1}}  
\newcommand{\eng}[1]{\begin{english}#1\end{english}}
% Подчиненные счетчики в окружениях http://old.kpfu.ru/journals/izv_vuz/arch/sample1251.tex
\newcounter{cTheorem} 
\newcounter{cDefinition}
\newcounter{cConsequent}
\newcounter{cExample}
\newcounter{cLemma}
\newcounter{cConjecture}
\newtheorem{Theorem}{Теорема}[cTheorem]
\newtheorem{Definition}{Определение}[cDefinition]
\newtheorem{Consequent}{Следствие}[cConsequent]
\newtheorem{Example}{Пример}[cExample]
\newtheorem{Lemma}{Лемма}[cLemma]
\newtheorem{Conjecture}{Гипотеза}[cConjecture]

\renewcommand{\theTheorem}{\arabic{Theorem}}
\renewcommand{\theDefinition}{\arabic{Definition}}
\renewcommand{\theConsequent}{\arabic{Consequent}}
\renewcommand{\theExample}{\arabic{Example}}
\renewcommand{\theLemma}{\arabic{Lemma}}
\renewcommand{\theConjecture}{\arabic{Conjecture}}
%\makeatletter
\NewDocumentCommand{\Newline}{}{\text{\\}}
\newcommand{\sequence}[2]{\ensuremath \left(#1,\ \dots,\ #2\right)}

\definecolor{mygreen}{rgb}{0,0.6,0}
\definecolor{mygray}{rgb}{0.5,0.5,0.5}
\definecolor{mymauve}{rgb}{0.58,0,0.82}
\renewcommand{\listalgorithmname}{Список алгоритмов}
\floatname{algorithm}{Листинг}
\renewcommand{\lstlistingname}{Листинг}
\renewcommand{\thealgorithm}{\arabic{algorithm}}

\newcommand{\refAlgo}[1]{(листинг \ref{#1})}
\newcommand{\refImage}[1]{(рисунок \ref{#1})}

\renewcommand{\theenumi}{\arabic{enumi})}% Меняем везде перечисления на цифра.цифра	
\renewcommand{\labelenumi}{\arabic{enumi})}% Меняем везде перечисления на цифра.цифра
\renewcommand{\theenumii}{\arabic{enumii}}% Меняем везде перечисления на цифра.цифра
\renewcommand{\labelenumii}{(\arabic{enumii})}% Меняем везде перечисления на цифра.цифра
\renewcommand{\theenumiii}{\roman{enumiii}}% Меняем везде перечисления на цифра.цифра
\renewcommand{\labelenumiii}{(\roman{enumiii})}% Меняем везде перечисления на цифра.цифра
%\newfontfamily\AnkaCoder[Path=src/fonts/]{AnkaCoder-r.ttf}
\renewcommand{\labelitemi}{---}
\renewcommand{\labelitemii}{---}

%\usepackage{courier}

\lstdefinelanguage{Refal}{
	alsodigit = {.,<,>},
	morekeywords = [1]{$ENTRY},
	morekeywords = [2]{Go, Put, Get, Open, Close, Arg, Add, Sub, Mul, Div, Symb, Explode, Implode},
	%keyword4
	morekeywords = [3]{<,>},
	%keyword5
	morekeywords = [4]{e.,t.,s.},
	sensitive = true,
	morecomment = [l]{*},
	morecomment = [s]{/*}{*/},
	commentstyle = \color{mygreen},
	morestring = [b]",
	morestring = [b]',
	stringstyle = \color{purple}
}

\makeatletter
\def\p@subsection{}
\def\p@subsubsection{\thesection\,\thesubsection\,}
\makeatother
\newcommand{\prog}[1]{{\ttfamily\small#1}}
\lstset{ %
	backgroundcolor=\color{white},   % choose the background color; you must add \usepackage{color} or \usepackage{xcolor}
	basicstyle=\ttfamily\footnotesize, 
	%basicstyle=\footnotesize\AnkaCoder,        % the size of the fonts that are used for the code
	breakatwhitespace=false,         % sets if automatic breaks shoulbd only happen at whitespace
	breaklines=true,                 % sets automatic line breaking
	captionpos=top,                    % sets the caption-position to bottom
	commentstyle=\color{mygreen},    % comment style
	deletekeywords={...},            % if you want to delete keywords from the given language
	escapeinside={\%*}{*)},          % if you want to add LaTeX within your code
	extendedchars=true,              % lets you use non-ASCII characters; for 8-bits encodings only, does not work with UTF-8
	inputencoding=utf8,
	frame=single,                    % adds a frame around the code
	keepspaces=true,                 % keeps spaces in text, useful for keeping indentation of code (possibly needs columns=flexible)
	keywordstyle=\bf,       % keyword style
	language=Refal,                    % the language of the code
	morekeywords={<,>,$ENTRY,Go,Arg, Open, Close, e., s., t., Get, Put}, 
	% if you want to add more keywords to the set
	numbers=left,                    % where to put the line-numbers; possible values are (none, left, right)
	numbersep=5pt,                   % how far the line-numbers are from the code
	xleftmargin=25pt,
	xrightmargin=25pt,
	numberstyle=\small\color{black}, % the style that is used for the line-numbers
	rulecolor=\color{black},         % if not set, the frame-color may be changed on line-breaks within not-black text (e.g. comments (green here))
	showspaces=false,                % show spaces everywhere adding particular underscores; it overrides 'showstringspaces'
	showstringspaces=false,          % underline spaces within strings only
	showtabs=false,                  % show tabs within strings adding particular underscores
	stepnumber=1,                    % the step between two line-numbers. If it's 1, each line will be numbered
	stringstyle=\color{mymauve},     % string literal style
	tabsize=8,                       % sets default tabsize to 8 spaces
	title=\lstname                   % show the filename of files included with \lstinputlisting; also try caption instead of title
}
\newcommand{\anonsection}[1]{\cleardoublepage
	\phantomsection
	\addcontentsline{toc}{section}{\protect\numberline{}#1}
	\section*{#1}\vspace*{2.5ex} % По госту положены 3 пустые строки после заголовка ненумеруемого раздела
}
\newcommand{\sectionbreak}{\clearpage}
\renewcommand{\sectionfont}{\normalsize} % Сбиваем стиль оглавления в стандартный
\renewcommand{\cftsecleader}{\cftdotfill{\cftdotsep}} % Точки в оглавлении напротив разделов

\renewcommand{\cftsecfont}{\normalfont\large} % Переключение на times в содержании
\renewcommand{\cftsubsecfont}{\normalfont\large} % Переключение на times в содержании

\setlength{\cftsecindent}{0pt}% Убираем отступы в содержании для \section
\setlength{\cftsubsecindent}{0pt}% Убираем отступы в содержании для \subsection
\setlength{\cftsubsubsecindent}{0pt}% Убираем отступы в содержании для \subsubsection

\usepackage{caption} 
%\captionsetup[table]{justification=raggedleft} 
%\captionsetup[figure]{justification=centering,labelsep=endash}
\captionsetup[listing]{justification=centering,labelsep=emdash}
\usepackage{amsmath}    % \bar    (матрицы и проч. ...)
\usepackage{amsfonts}   % \mathbb (символ для множества действительных чисел и проч. ...)
\usepackage{mathtools}  % \abs, \norm
\DeclarePairedDelimiter\abs{\lvert}{\rvert} % операция модуля
\DeclarePairedDelimiter\norm{\lVert}{\rVert} % операция нормы
\DeclareTextCommandDefault{\textvisiblespace}{%
	\mbox{\kern.06em\vrule \@height.3ex}%
	\vbox{\hrule \@width.3em}%
	\hbox{\vrule \@height.3ex}}    
\newsavebox{\spacebox}
\begin{lrbox}{\spacebox}
	\verb*! !
\end{lrbox}
\newcommand{\aspace}{\usebox{\spacebox}}
\DeclareTotalCounter{listing}

\makeatletter
\renewcommand*{\p@subsubsection}{}
\makeatother

\newenvironment{longlisting}{\captionsetup{type=listing}}{}
