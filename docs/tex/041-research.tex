\section{Исследовательская часть}

В текущем разделе приведены результаты нагрузочного тестирования разработанного программного обеспечения с использованием Apache Benchmarks.
Также приведены результаты сравнения с тестированием NGINX.
Сравнение с сервером NGINX проводится по причине его первенства среди рассмотренных существующих решений.
Описываются характеристики машины, на которой производится тестирование.

\subsection{Описание тестирования}

Нагрузочное тестирование осуществляется с использованием Apache Benchmarks.
Время обработки замерялось от 100 до 1000 запросов с шагом 100 при подключении 5, 50, 100 клиентов.
Тестировался разработанный сервер, а также NGINX.

Технические характеристики машины, на которой проводилось тестирование:
\begin{itemize}
	\item операционная система --- Ubuntu 24.04.1 LTS;
	\item процессор --- 13th Gen Intel® Core™ i7-1360P × 16;
	\item оперативная память --- 16,0 Гб LPDDR5 с тактовой частотой 4800 Гц;
	\item графическая карта --- Intel® Iris® Xe Graphics (RPL-P).
\end{itemize}

Тестирование выполнялось на ноутбуке, являющимся подключенным к сети электропитания.
Во время тестирования ноутбук был нагружен только системой тестирования (работающим приложением) и окружением операционной системы.

Результаты тестирования для 5, 50 и 100 клиентов приведены на рисунках \ref{img:5}, \ref{img:50} и \ref{img:100} соответственно.

\begin{figure}[!htb]\centering
	\includegraphics[width=0.3\textwidth]{../img/5.png}
	\caption{Результаты тестирования (5)}
	\label{img:5}
\end{figure}

Результаты тестирования показывают, что разработанная программа обрабатывает запросы на статическую информацию быстрее, чем NGINX.
Разница во времени обработки тем больше, чем меньше количество подключённых клиентов: при 1000 запросах и 5 клиентах разработанный сервер работает в 7 раз быстрее, а при 50 клиентах — в 3 раза.
Это объясняется тем, что при тестировании количество запросов оставалось постоянным (от 100 до 1000 с шагом 100), поэтому суммарное число запросов на одного клиента уменьшалось с ростом числа клиентов.

\begin{figure}[!htb]\centering
	\includegraphics[width=0.6\textwidth]{../img/59.png}
	\caption{Результаты тестирования (5)}
	\label{img:50}
\end{figure}

\begin{figure}[!htb]\centering
	\includegraphics[width=0.6\textwidth]{../img/100.png}
	\caption{Результаты тестирования (5)}
	\label{img:100}
\end{figure}
