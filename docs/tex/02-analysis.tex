\section{Аналитический раздел}

\subsection{Постановка задачи}

В соответствии с заданием на курсовую работу необходимо разработать драйвер для изменения яркости и цветовой температуры дисплея с использованием USB-мыши.

\subsection{Особенности шины USB}

Типы устройств:

b — блок-ориентированные устройства;
c — байт-ориентированные устройства (символьные);

u — не буфферизованное байт-ориентированное устройство; p — именованный канал.

Для хранения номеров устройств, в ядре используется специальный тип dev\_t. Начиная с ядра 2.6.0 — dev\_t 32-х разрядный, 12 отведены для старшего номера (все устройства делятся на наборы, которые определяются старшим номером: , например, все SCSI-диски имеют старший номер 8), 20 для младшего (младший номер конкретизирует устройство). Для получения старшей и младшей части используются макросы MAJOR и MINOR.

%Драйвер — программа или часть кода ядра, которая предназначена для управления конкретным внешним устройством. Обычно содержат набор команд, специфичных для этого устройства. Необходимость драйверов этим и объясняется: каждое устройство воспринимает только специальный набор команд (и для каждого устройства он различен).

В Linux драйверы трех типов:

Встроенные в ядро. Соответствующие устройства распознаются системой автоматически, нужны для поддержки, например, монтирования корневой ФС и запуска ПК (материнская плата и т.д.)
Реализованные, как загружаемые модули ядра. Используются для управления устройствами (звуковые и сетевые карты, SCSI адаптеры)
Код поделен между ядром и специальной утилитой, управляющей устройством. Например, принтер: ядро осуществляет взаимодействие с параллельным портом, а формирование управляющих сигналов для принтера осуществляет демон печати lpd.

Схема взаимодействия прикладных программ с аппаратной частью компьютера (в Linux): устройство ‹-› ядро ‹-› специальный файл устройства ‹-› программа пользователя

Любому устройству, зарегистрировавшемуся в системе, выделяется специальный дескриптор в виде структуры struct device. Структура struct device\_driver представляет универсальную модель драйвера устройства, которая отслеживает все драйверы, известные системе.

usb\_register\_dev

Устройства USB могут являться хабами, функциями или их комбинацией.

Хаб (Hub) - обеспечивает дополнительные точки подключения устройств к шине
Функции представляют собой устройства, способные передавать или принимать данные или управляющую информацию по шине. Функции USB предоставляют системе дополнительные возможности, например подключение акустической системы, мыши и т.п.

USB-шина является шиной, в которой имеется один мастер - host-контроллер (который еще называют root hub). USB-устройства всегда отвечают на запросы host-контроллера (usb-устройства никогда не могут посылать информацию самостоятельно; host-контроллер формирует запросы, а устройства отвечают).

% схема дерева

% схема архитектуры общего драйвера

Запросы имеют направление:

IN - хост отправляет запрос на прием данных
OUT - хост отсылает данные устройству \cite{corbet2005linux}

Конечные точки (endpoints). Это базовый объект связи интерфейса USB. Устройство может иметь до 16 конечных точек, нумерация начинается с 0 и заканчивается 15. Каждая конечная точка может включать в себя два буфера (адреса): входной и выходной. То есть устройство может обладать 32 адресами конечных точек. Каждая USB-функция должна содержать как минимум одну (нулевую) конечную точку с входным и выходным буфером.

Каналы (pipes). Хост определяет каналы, которые связаны с конечными точками функции. В отличие от конечной точки, которая имеет физическую сущность в нашем мире, канал является всего лишь логической концепцией, правилом. После установки канала, становится определенным и тип передачи данных, который он поддерживает.

Данные отправляются и принимаются посредством передач или сообщений, которые состоят из ряда транзакций, каждая из которых состоит из пакетов.

Существует 4 типа передач:

control — передача типа control является двунаправленным и предназначен для обмена с устройством короткими пакетами типа «вопрос-ответ». Обеспечивает гарантированную доставку данных.
isochronous — изохронный канал имеет гарантированную пропускную способность (N пакетов за один период шины) и обеспечивает непрерывную передачу данных.
interrupt — канал прерывания позволяет доставлять короткие пакеты без гарантии доставки и без подтверждений приема, но с гарантией времени доставки – пакет будет доставлен не позже, чем через N миллисекунд.
bulk — поточная или сплошная передача используется устройствами, отправляющими и принимающими большое количество данных. Канал дает гарантию доставки каждого пакета.

\subsection{USB драйвер}

% виды драйверов 

Структура USB-драйвера struct usb\_driver (Основные точки входа: probe, disconnect, suspend, resume).

% struct usb\_driver

id\_table используется для горячего подключения. Он содержит набор дескрипторов, и специализированные данные могут быть связаны с каждой записью. Эта таблица используется поддержкой горячего подключения как в режиме пользователя, так и в режиме ядра.

% struct id\_table ? what yes

usb\_alloc\_urb, usb\_fill\_int\_urb, usb\_submit\_urb

\subsection{Способы изменения яркости, цветовой температуры}

% intel\_backlight\_set\_acpi
% sysfs

% gnome

\subsection{Тип программного обеспечения}

драйвер + демон