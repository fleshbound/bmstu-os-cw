\anonsection{ВВЕДЕНИЕ}

Одной из важнейших задач операционной системы является организация работы с устройствами ввода-вывода.
Наиболее распространенными представителями таких устройств являются стандартные компьютерные мыши, клавиатуры, джойстики, мониторы; они позволяют пользователю интерактивно взаимодействовать с компьютерной системой.

Управление внешними устройствами драйверами, изменение их поведения осуществляется написанием драйверов.
Ключевыми характеристиками дисплея компьютера являются яркость и цветовая температура (теплота).
Чтобы задать необходимые пользователю (в частности, безопасные для его зрения) значения яркости и теплоты, целесообразно использовать устройство, позволяющее одновременно нажимать на клавиши и осуществлять скроллинг.
Одним из таких устройств является USB-мышь.
